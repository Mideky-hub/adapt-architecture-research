% Options for packages loaded elsewhere
% Options for packages loaded elsewhere
\PassOptionsToPackage{unicode}{hyperref}
\PassOptionsToPackage{hyphens}{url}
\PassOptionsToPackage{dvipsnames,svgnames,x11names}{xcolor}
%
\documentclass[
  letterpaper,
  DIV=11,
  numbers=noendperiod]{scrartcl}
\usepackage{xcolor}
\usepackage{amsmath,amssymb}
\setcounter{secnumdepth}{-\maxdimen} % remove section numbering
\usepackage{iftex}
\ifPDFTeX
  \usepackage[T1]{fontenc}
  \usepackage[utf8]{inputenc}
  \usepackage{textcomp} % provide euro and other symbols
\else % if luatex or xetex
  \usepackage{unicode-math} % this also loads fontspec
  \defaultfontfeatures{Scale=MatchLowercase}
  \defaultfontfeatures[\rmfamily]{Ligatures=TeX,Scale=1}
\fi
\usepackage{lmodern}
\ifPDFTeX\else
  % xetex/luatex font selection
\fi
% Use upquote if available, for straight quotes in verbatim environments
\IfFileExists{upquote.sty}{\usepackage{upquote}}{}
\IfFileExists{microtype.sty}{% use microtype if available
  \usepackage[]{microtype}
  \UseMicrotypeSet[protrusion]{basicmath} % disable protrusion for tt fonts
}{}
\makeatletter
\@ifundefined{KOMAClassName}{% if non-KOMA class
  \IfFileExists{parskip.sty}{%
    \usepackage{parskip}
  }{% else
    \setlength{\parindent}{0pt}
    \setlength{\parskip}{6pt plus 2pt minus 1pt}}
}{% if KOMA class
  \KOMAoptions{parskip=half}}
\makeatother
% Make \paragraph and \subparagraph free-standing
\makeatletter
\ifx\paragraph\undefined\else
  \let\oldparagraph\paragraph
  \renewcommand{\paragraph}{
    \@ifstar
      \xxxParagraphStar
      \xxxParagraphNoStar
  }
  \newcommand{\xxxParagraphStar}[1]{\oldparagraph*{#1}\mbox{}}
  \newcommand{\xxxParagraphNoStar}[1]{\oldparagraph{#1}\mbox{}}
\fi
\ifx\subparagraph\undefined\else
  \let\oldsubparagraph\subparagraph
  \renewcommand{\subparagraph}{
    \@ifstar
      \xxxSubParagraphStar
      \xxxSubParagraphNoStar
  }
  \newcommand{\xxxSubParagraphStar}[1]{\oldsubparagraph*{#1}\mbox{}}
  \newcommand{\xxxSubParagraphNoStar}[1]{\oldsubparagraph{#1}\mbox{}}
\fi
\makeatother


\usepackage{longtable,booktabs,array}
\usepackage{calc} % for calculating minipage widths
% Correct order of tables after \paragraph or \subparagraph
\usepackage{etoolbox}
\makeatletter
\patchcmd\longtable{\par}{\if@noskipsec\mbox{}\fi\par}{}{}
\makeatother
% Allow footnotes in longtable head/foot
\IfFileExists{footnotehyper.sty}{\usepackage{footnotehyper}}{\usepackage{footnote}}
\makesavenoteenv{longtable}
\usepackage{graphicx}
\makeatletter
\newsavebox\pandoc@box
\newcommand*\pandocbounded[1]{% scales image to fit in text height/width
  \sbox\pandoc@box{#1}%
  \Gscale@div\@tempa{\textheight}{\dimexpr\ht\pandoc@box+\dp\pandoc@box\relax}%
  \Gscale@div\@tempb{\linewidth}{\wd\pandoc@box}%
  \ifdim\@tempb\p@<\@tempa\p@\let\@tempa\@tempb\fi% select the smaller of both
  \ifdim\@tempa\p@<\p@\scalebox{\@tempa}{\usebox\pandoc@box}%
  \else\usebox{\pandoc@box}%
  \fi%
}
% Set default figure placement to htbp
\def\fps@figure{htbp}
\makeatother





\setlength{\emergencystretch}{3em} % prevent overfull lines

\providecommand{\tightlist}{%
  \setlength{\itemsep}{0pt}\setlength{\parskip}{0pt}}



 


\KOMAoption{captions}{tableheading}
\makeatletter
\@ifpackageloaded{caption}{}{\usepackage{caption}}
\AtBeginDocument{%
\ifdefined\contentsname
  \renewcommand*\contentsname{Table of contents}
\else
  \newcommand\contentsname{Table of contents}
\fi
\ifdefined\listfigurename
  \renewcommand*\listfigurename{List of Figures}
\else
  \newcommand\listfigurename{List of Figures}
\fi
\ifdefined\listtablename
  \renewcommand*\listtablename{List of Tables}
\else
  \newcommand\listtablename{List of Tables}
\fi
\ifdefined\figurename
  \renewcommand*\figurename{Figure}
\else
  \newcommand\figurename{Figure}
\fi
\ifdefined\tablename
  \renewcommand*\tablename{Table}
\else
  \newcommand\tablename{Table}
\fi
}
\@ifpackageloaded{float}{}{\usepackage{float}}
\floatstyle{ruled}
\@ifundefined{c@chapter}{\newfloat{codelisting}{h}{lop}}{\newfloat{codelisting}{h}{lop}[chapter]}
\floatname{codelisting}{Listing}
\newcommand*\listoflistings{\listof{codelisting}{List of Listings}}
\makeatother
\makeatletter
\makeatother
\makeatletter
\@ifpackageloaded{caption}{}{\usepackage{caption}}
\@ifpackageloaded{subcaption}{}{\usepackage{subcaption}}
\makeatother
\usepackage{bookmark}
\IfFileExists{xurl.sty}{\usepackage{xurl}}{} % add URL line breaks if available
\urlstyle{same}
\hypersetup{
  colorlinks=true,
  linkcolor={blue},
  filecolor={Maroon},
  citecolor={Blue},
  urlcolor={Blue},
  pdfcreator={LaTeX via pandoc}}


\author{}
\date{}
\begin{document}


\section{Design and Development of cloud-based microservice architecture
for resilient, performant and scalable applications using ADAPT Design
Patterns}\label{design-and-development-of-cloud-based-microservice-architecture-for-resilient-performant-and-scalable-applications-using-adapt-design-patterns}

Mr.~Mathieu KERBEL

Senior Software Engineer \& Cloud Solutions Architect Creator of the
ADAPT Architecture Principle Paris, Île-de-France, France
mathieukerbel@gmail.com

\begin{figure}[H]

{\centering \pandocbounded{\includegraphics[keepaspectratio]{README_files/mediabag/Architecture-ADAPT-b.pdf}}

}

\caption{ADAPT Architecture}

\end{figure}%

\subsection{Abstract}\label{abstract}

Microservices have revolutionized software architecture by promoting
modularity, scalability, interoperability and resilience. However,
traditional design principles like SOLID have shown limitations in
addressing the complexities of microservices architectures, more
particularly in cloud-based environments. This research paper explores
the application of the ADAPT (Asynchronous, Domain-distributed,
Abstraction-moderated, Piloted through events, Transparent and
observable) architecture principles and design patterns to develop a
robust microservices architecture for cloud-based applications.

\subsection{Introduction}\label{introduction}

The history of software architecture is continous struggl against
entropy and complexity. As applications grow in size and functionality,
the era of monolithic architectures with the logical architecture (how
the code is organized) and thge physical architecture (how the code is
deployed) tightly coupled together has shown its limitations.
Microservices architecture emerged as a solution to these challenges,
offering a way to break down applications into smaller, independent
services that can be developed, deployed, and scaled independently.

If the SOLID principles

/// continue once results has been achieved.

\subsection{Literature Review}\label{literature-review}

\subsection{Methodology}\label{methodology}

\subsubsection{Event-Driven Architecture and DX-based
approach}\label{event-driven-architecture-and-dx-based-approach}

\subsubsection{ADAPT Principles and Design
Patterns}\label{adapt-principles-and-design-patterns}

\paragraph{Asynchronous Communication}\label{asynchronous-communication}

\begin{quote}
``Services must be decoupled, communicating through events rather than
direct calls to build resilient and scalable systems.''
\end{quote}

First principle of ADAPT architecture {[}1{]}{[}2{]} focuses on
asynchronous communication between microservices to enhance scalability
and resilience. This section explores various asynchronous communication
patterns such as message queues, event streaming, and publish--subscribe
models.

\subparagraph{Message queues, event streaming, and publish--subscribe
models using Apache
Kafka}\label{message-queues-event-streaming-and-publishsubscribe-models-using-apache-kafka}

Kafka is a distributed event streaming platform that enables
asynchronous communication between microservices. It allows services to
publish and subscribe to streams of records, facilitating decoupled
interactions. As there are multiple ways to manage message queues, Kafka
is one of the most popular solutions for building event-driven
architectures, therefore it is used as an example in this section.

Figure 1: Kafka Cluster in an Emitting and Receiving System Component

A few rikes can benefit from using Kafka in microservices architecture
include :

\begin{itemize}
\item
  \textbf{Decoupling services}: Kafka allows services to communicate
  without direct dependencies, enhancing flexibility and
  maintainability.
\item
  \textbf{Scalability}: Kafka's distributed architecture supports high
  throughput and can handle large volumes of data, its axis of
  scalability is horizontal due to its distributed nature. As data
  expand across multiple brokers, Kafka can efficiently manage increased
  loads by adding more brokers to the cluster.
\item
  \textbf{Resilience \& Fault Tolerance}: Kafka's fault-tolerant design
  ensures that messages are not lost, even in the event of service
  failures. It achieves this through data replication across multiple
  brokers and automatic failover mechanisms.
\item
  \textbf{Real-time Data Processing}: Kafka enables real-time data
  streaming and processing, which is essential for applications
  requiring immediate insights and actions.
\item
  \textbf{Event Sourcing}: Kafka can be used to implement event sourcing
  patterns, where state changes are logged as a sequence of events,
  allowing for better traceability and recovery.
\end{itemize}

\paragraph{Domain-Distributed-Driven
Design}\label{domain-distributed-driven-design}

\begin{quote}
``The architecture must mirror the business domain, with services
organized around business capabilities, not technical layers.''
\end{quote}

The second principle of ADAPT architecture {[}1{]}{[}2{]} emphasizes the
importance of aligning microservices with business domains. This section
discusses Domain-Driven Design (DDD) concepts such as bounded contexts,
aggregates, and domain events to structure services around business
capabilities.

The ADAPT principle redefines the single responsibility principle from
SOLID with a business capability having the complete ownership of its
data and logic within a bounded context. This approach ensures that each
microservice is responsible for a specific business function, avoinding
some common pitfalls of microservices that going against the atomicity
of business capabilities.

\paragraph{Abstraction Moderation}\label{abstraction-moderation}

\paragraph{Piloted through Events and
Configurations}\label{piloted-through-events-and-configurations}

\paragraph{Transparency and Observability through
Contacts}\label{transparency-and-observability-through-contacts}

\subsubsection{Implementation}\label{implementation}

\subsubsection{Case Study: E-commerce
Application}\label{case-study-e-commerce-application}

To demonstrate the practical application of the ADAPT principles, this
chapter details the migration of a reference legacy system to a
cloud-based microservices architecture using ADAPT design patterns.

\paragraph{System Architecture}\label{system-architecture}

The current baseline architecture of the e-commerce application is
structured as a typical legacy e-commerce application with a
\textbf{\emph{layered architecture}} with a \textbf{Controller Layer},
\textbf{Service Layer} and a \textbf{Repository Layer}. The application
is \emph{monolithic}, with all components \textbf{\emph{tightly
coupled}} together.

\paragraph{Technology Stack}\label{technology-stack}

The technical stack includes a relational database (\emph{PostgreSQL}),
a backend developed in \emph{Java} with \emph{Spring Boot} framework.
The current service coupling is high, the \textbf{Order Service} has
compile-time dependencies on the \textbf{Inventory Service} and the
\textbf{Payment Service}. The Data is managed through a single database
schema shared across all services with the \textbf{Billing Service}
directly accessing the \textbf{User Service} database tables.

\subsubsection{Benchmarking}\label{benchmarking}

The validity of the ADAPT Architecture Principles and Design Patterns is
assessed using a suite of quantitative metrics derived from recent
software engineering litterature. Those metrics replace vague notions of
``clean code'' and ``clean architecture'' with measurable criteria and
structural attributes.

\paragraph{What to measure?}\label{what-to-measure}

Derived from Panichella et al.~(2021) {[}15{]}, the structural coupling
measures the senity of dependencies between microservices. Lower
coupling values indicate better modularity and independence among
services. It calcultaed as follows:

\[
SC(s_i) = \frac{\sum_{j \in S; j \not= i} dep(s_i, s_j)}{|S|}
\]

Where \(dep(s_i, s_j)\) indicated a dependency from service \(i\) to
service \(j\), and \(|S|\) is the total number of services in the
system. If \(SC(s_i) = n\) , it means that service \(s_i\) depends on
\(n\) other services.

With microservices architectures, when our services are too granular and
too many, there's a nano-service anti-pattern that can appear, we can
measure the Weighted Service Interface Count (WSIC) and the Service
Interface Data Cohesion (SIDC).

The WSIC metric {[}20{]} measures API surface area and number of
interfaces a service exposes, weighted by their complexity.

The SIDC is the metric that quantifie the cohesion of a service based on
the cohesiveness of the operations exposed in its interface. It's
defined as follows:

\[
\begin{aligned}
S &= \operatorname{SOp}(si_s)=\{o_1,\dots,o_n\},\quad n=|S|.\\
\text{For }1\le i<j\le n:&\\
\quad I_{\mathrm{param}}(o_i,o_j)&=
\begin{cases}
1,& \operatorname{ParamTypes}(o_i)\cap\operatorname{ParamTypes}(o_j)\neq\varnothing,\\
0,&\text{otherwise},
\end{cases}\\
\quad I_{\mathrm{ret}}(o_i,o_j)&=
\begin{cases}
1,& \operatorname{Ret}(o_i)=\operatorname{Ret}(o_j),\\
0,&\text{otherwise}.
\end{cases}\\
P &= \binom{n}{2}\quad(\text{number of unordered operation pairs}).\\
\text{Then}\qquad SIDC(s) &=
\begin{cases}
\displaystyle\frac{\sum_{1\le i<j\le n}\big(I_{\mathrm{param}}(o_i,o_j)+I_{\mathrm{ret}}(o_i,o_j)\big)}{2P}, & P>0,\\[8pt]
0, & P=0.
\end{cases}
\end{aligned}
\]

Beyond static code analysis, we also have to considers the operational
metrics defined by the DORA (DevOps Research and Assessment) research
program, which are widely accepted as key indicators of software
delivery performance. Those metrics are the deployment frequency, the
change failure rate, the lead time for changes and the mean time to
recovery (MTTR).

\paragraph{Performance Testing}\label{performance-testing}

\subsection{Results and Discussion}\label{results-and-discussion}

\subsubsection{Scalability}\label{scalability}

\subsubsection{Resilience}\label{resilience}

\subsubsection{Performance}\label{performance}

\subsubsection{Maintainability}\label{maintainability}

\subsubsection{Developer Experience}\label{developer-experience}

\subsubsection{Future Work}\label{future-work}

\subsection{Conclusion}\label{conclusion}

\subsubsection{Acknowledgements}\label{acknowledgements}

\subsubsection{Weaknesses}\label{weaknesses}

\subsection{References}\label{references}

{[}1{]} Kerbel, M. (2025). SOLID is killing the micro-services industry
--- ADAPT saves it. Medium.
https://mathieukerbel.medium.com/solid-is-killing-the-micro-services-industry-adapt-saves-it-7f3f5f4f4f4f

{[}2{]} Kerbel, M. (2025). The ADAPT Architecture Manifesto. GitHub.
https://github.com/Mideky-hub/adapt-architecture-principle/blob/main/README.md

{[}3{]} Rasheedh, J. A., \& Saradha, S. (2022). Design and Development
of Resilient Microservices Architecture for Cloud Based Applications
Using Hybrid Design Patterns. Indian Journal of Computer Science and
Engineering, 13(2), 365--378. PDF:
https://www.researchgate.net/profile/J-Rasheedh/publication/360132110\_Design\_and\_Development\_of\_Resilient\_Microservices\_Architecture\_for\_Cloud\_Based\_Applications\_using\_Hybrid\_Design\_Patterns/links/6275381eb1ad9f66c8a72ce1/Design-and-Development-of-Resilient-Microservices-Architecture-for-Cloud-Based-Applications-using-Hybrid-Design-Patterns.pdf

{[}4{]} Bailo, D., Jeffery, K. G., Spinuso, A., \& Fiameni, G. (2015).
Interoperability Oriented Architecture: The Approach of EPOS for Solid
Earth e-Infrastructures. In 2015 IEEE 11th International Conference on
e-Science, Munich, Germany. doi:10.1109/eScience.2015.22

{[}5{]} Di Francesco, P., Lago, P., \& Malavolta, I. (2019).
Architecting with microservices: A systematic mapping study. Journal of
Systems and Software, 150, 77--97.

{[}6{]} Balalaie, A., Heydarnoori, A., \& Jamshidi, P. (2016).
Microservices architecture enables DevOps: Migration to a cloud-native
architecture. IEEE Software, 33(3), 42--52.

{[}7{]} Newman, S. (2015). Building Microservices: Designing
Fine-Grained Systems. O'Reilly Media.

{[}8{]} Taibi, D., Lenarduzzi, V., \& Pahl, C. (2020). Microservices
Anti-patterns: A Taxonomy. In Microservices (pp.~111--128). Springer,
Cham.

{[}9{]} Soldani, J., Tamburri, D. A., \& Van Den Heuvel, W. J. (2018).
The pains and gains of microservices: A systematic grey literature
review. Journal of Systems and Software, 146, 215--232.

{[}10{]} Márquez, G., \& Astudillo, H. (2018). Actual use of
architectural patterns in microservices-based open source projects. In
2018 25th Asia-Pacific Software Engineering Conference (APSEC)
(pp.~31--40). IEEE.

{[}11{]} Lewis, J., \& Fowler, M. (2014). Microservices --- A definition
of this new architectural term. martinfowler.com.
https://martinfowler.com/articles/microservices.html

{[}12{]} Ibryam, B., \& Huß, R. (2023). Kubernetes Patterns: Reusable
Elements for Designing Cloud-Native Applications (2nd ed.). O'Reilly
Media.

{[}13{]} Burns, B. (2018). Designing Distributed Systems: Patterns and
Paradigms for Scalable, Reliable Services. O'Reilly Media.

{[}14{]} Ponugoti, M. (2025). Cloud-Native Platform Engineering:
Scalable Design Patterns for Global Enterprise Resilience. Journal of
Computer Science and Technology Studies, 7(8), 48--59.

{[}15{]} Panichella, S., Rahman, M.I., \& Taibi, D. (2021). Structural
Coupling for Microservices. International Conference on Cloud Computing
and Services Science.

{[}16{]} Merson, P. (2020). Principles for microservice design: Think
IDEALS, rather than SOLID. InfoQ.
https://www.infoq.com/articles/microservices-design-ideals/

{[}17{]} Martin, R. C. (2017). Clean Architecture: A Craftsman's Guide
to Software Structure and Design. Prentice Hall.

{[}18{]} Millett, S., \& Tune, N. (2015). Patterns, Principles, and
Practices of Domain-Driven Design. Wrox.

{[}19{]} Richardson, C. (2019). Pattern: API Gateway. microservices.io.
https://microservices.io/patterns/apigateway.html

{[}20{]} Restful-ma, C. (2019). Weighted Service Interface Count (WSIC).
https://github.com/restful-ma/rama-cli/blob/master/docs/metrics/WeightedServiceInterfaceCount.md

{[}21{]} Mateus Gabi Moreira and Breno Bernard Nicolau De França. 2022.
Analysis of Microservice Evolution using Cohesion Metrics. In
Proceedings of the 16th Brazilian Symposium on Software Components,
Architectures, and Reuse (SBCARS '22). Association for Computing
Machinery, New York, NY, USA, 40--49.
https://doi.org/10.1145/3559712.3559716

{[}22{]} Yiming Z., Tiziano D. M., \& Justus B. (2025). How Does
Microservice Granularity Impact Energy Consumption and Performance? A
Controlled Experiment. https://arxiv.org/html/2502.00482v1

{[}23{]} Hirzalla, M., Cleland-Huang, J., Arsanjani, A. (2009). A
Metrics Suite for Evaluating Flexibility and Complexity in Service
Oriented Architectures. In: Feuerlicht, G., Lamersdorf, W. (eds)
Service-Oriented Computing -- ICSOC 2008 Workshops. ICSOC 2008. Lecture
Notes in Computer Science, vol 5472. Springer, Berlin, Heidelberg.
https://doi.org/10.1007/978-3-642-01247-1\_5




\end{document}
